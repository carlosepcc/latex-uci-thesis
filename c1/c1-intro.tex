\section[Introducción]{Introducción del capítulo}
En el presente capítulo se engloban aspectos relacionados con el objeto de estudio definido para el problema planteado. El análisis de algunas metodologías, procedimientos, herramientas existentes para el desarrollo de sistemas web y la observación de aplicaciones homólogas; permitirá la selección de las tecnologías adecuadas para el desarrollo del Sistema de Gestión par el Proceso de Comisión Disciplinaria y contar con un análisis de sistemas existentes que realizan funcionalidades similares.
Sobre la base de los elementos expuestos anteriormente se formula el siguiente problema de investigación: ¿Cómo contribuir a la agilización del proceso de comisión disciplinaria en la facultad 4? Para la realización de la investigación se define como objeto de estudio: el proceso de comisión disciplinaria qe se lleva a cabo en la facultad 4.
Para dar solución al problema planteado, se define como objetivo general: desarrollar un sistema para el la gestión del proceso de comisión disciplinaria en la Facultad 4 de la \ac{uci}

Para dar cumplimiento al objetivo general antes mencionado, se dará cumplimiento a los siguientes objetivos específicos:

\begin{enumerate}
	\item Describir el estado actual de las herramientas dirigidas a la gestión de procesos disciplinarios en casas de altos estudios.
	\item Definir las tecnologías, herramientas y metodología a utilizar en la implementación de un sistema de gestión para el proceso de comisión disciplinaria en la facultad 4.	\item Diseñar las funcionalidades sistema de gestión para el proceso de comisión disciplinaria en la facultad 4.
	\item Implementar y validar las funcionalidades del sistema de gestión para el proceso de comisión disciplinaria en la facultad 4
\end{enumerate}

\textbf{Hipótesis:}
Con el desarrollo de un sistema de gestión para el proceso de comisión disciplinaria en la facultad 4 se contribuirá a la mejora del proceso de apoyo a la toma de decisiones.
Se define como \emph{variable independiente}: módulo de procesamiento estadístico de información y como  \emph{variable dependiente}: proceso de apoyo a la toma de decisiones.
