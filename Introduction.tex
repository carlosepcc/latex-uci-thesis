\chapter*{Introducción}
La creación de la Universidad de las Ciencias Informáticas (UCI) le permite a Cuba insertarse en el vertiginoso desarrollo del software y formar profesionales con alto nivel competitivo en el campo de la informática. 
El presente trabajo tiene como objetivo desarrollar el Sistema para la Gestión del Proceso de Comisión Disciplinaria en la Facultad 4 de la Universidad de las Ciencias Informáticas, permitiendo la agilización del flujo de información durante el proceso. 
Teniendo en cuenta la situación problemática anteriormente descrita, se plantea como problema de investigación la siguiente interrogante: 
¿Cómo contribuir con informatización del proceso de comisión disciplinaria en la Facultad 4? 
Por lo cual el objeto de estudio es el proceso de comisión disciplinarial. 
Por tanto, el campo de acción está enmarcado en los procesos de gestión de resoluciones decanales, denuncias, casos disciplinarios y declaraciones en la Facultad 4. 
Se define como objetivo general: Desarrollar un sistema de gestión para el proceso de comisión disciplinaria en la Facultad 4. 
A partir del objetivo general se derivan los siguientes objetivos específicos: 
Analizar elementos teóricos y principales tendencias del desarrollo de sistemas en la actualidad, específicamente en el campo de sistemas de gestión de la información.
Definir las tecnologías y herramientas necesarias para el desarrollo de la propuesta de solución. 
Implementar las funcionalidades de la propuesta de solución. 
Evaluar la propuesta de solución. 
Como Hipótesis se plantea: El desarrollo de una aplicación informática de gestión para el proceso de comisión disciplinaria en la Facultad 4 contribuirá a la informatización del proceso de comisión disciplinaria en la Facultad 4. 
En la investigación se utilizaron los siguientes métodos científicos:
Métodos teóricos 
Histórico-lógico: Para el estudio del desarrollo y evolución de los diferentes sistemas de gestión de 
información similares, nacionales e internacionales, así como las herramientas y tecnologías para el desarrollo del software, entre ellos los lenguajes de programación, framework de desarrollo, metodologías y herramientas CASE.

Analítico-sintético: En la realización del análisis de la información empleada para la investigación, identificando así, conceptos, definiciones y avances acerca de los sistemas de gestión de información existentes. 
Modelación: Se utiliza en la modelación de los diagramas dentro de la metodología de desarrollo de software seleccionada para llevar a cabo la solución. 

Métodos Empíricos 
Observación: Se utilizó para obtener información de las necesidades existentes en la Facultad 4. 